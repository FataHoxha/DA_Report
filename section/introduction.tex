Raft is a consensus algorithm for managing a replicated log
proposed by Diego Ongaro and John Ousterhout~\cite{ongaro2014search}.
It is equivalent to the state of the art (multi-)Paxos~\cite{lamport1998part} in fault-tolerance and performance and
aims at replacing it thanks to its understandability.
Consensus algorithms tend to be complex due to the huge problem they try to solve,
and leaving space for interpretation, like Lamport
did when proposing Paxos via a tale~\cite{lamport1998part}, does not help.
The combination of Paxos' complexity and the way it was presented required
Lamport to release an alternative description~\cite{lamport2001paxos}.
Even if the author claims that ``The Paxos algorithm, when presented in plain
English, is very simple.'', the level of abstraction
with which Paxos was designed and its complex architecture do not map
to real world applications and complicate its implementations, which are sometimes
required to relax the model leading to unproved protocols~\cite{chandra2007paxos}.
The same argument holds for the students and researches.\\
The creator thus identified in \emph{understandability} its ``most important goal''~\cite{ongaro2014consensus}
while describing Raft, breaking it into smaller independant subproblems and cleanly addressing all pieces needed
for a pratical application.

What is the simplest way to present and try to understand an interaction based
protocol?\emph{Visualizing it!}

Ongaro created and uses a Raft simulator, Raft Scope~\cite{raftscope}, to introduce the protocol.
This tool is very useful to simulate easy scenarios and to create slides for presentations,
however it does not currently implement all the features of the protocol and lacks
some graphics cues which can be key in the presentation of many corner cases.

In our project we aim to complete the protocol implementation, integrate new
functionalities with the existing ones and improve the user interface where needed to
enhance transparency and thus make Raft even more understandable.
