Raft is a consensus algorithm for managing a replicated log
proposed by Ontaro and Ousterhout~\cite{ongaro2014search}.
It is powerful as the state of the art (multi-)Paxos~\cite{lamport1998part} and
aims at replacing it thanks to its understandability.
Consensus algorithm tend to be complex due to the problem they try to solve,
it is not worthless saying that leaving space for interpretation, like Lamport
did when proposed Paxos via a tale~\cite{lamport1998part}, does not help.
The combination of Paxos' complexity and the way it was presented required
indeed Lamport to release an alternative description~\cite{lamport2001paxos}.
Even if the author claims that ``The Paxos algorithm, when presented in plain
English, is very simple.'' it seams that the level of abstraction
with which Paxos was designed and its complex architecture which does not map
to the real world applications complicate its implementations, which sometimes
require to relax the model leading to unproved protocols~\cite{chandra2007paxos}.
The same argument holds for the students and researches.\\
Ongaro, thus, identified in the \emph{understandability} its ``most important goal''~\cite{ongaro2014consensus}.

What is the simplest way to present an try to understand and interaction based
protocol?\emph{Visualizing it!}

Ongaro uses and provides a Raft simulator, Raft Scope~\cite{raftscope}, to introduce the protocol.
Such tool is very useful for easy scenarios simulation and slides creation,
unfortunately it does not implement all the features of the protocol and lacks
some graphics functionalities which can be key in the presentation of
many corner cases.

In our project we aim at complete the protocol implementation, integrate the new
functionalities with the existing one and improve the graphics where needed to
enhance transparency and thus enable even more understanding.
