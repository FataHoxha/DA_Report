Raft is a consensus algorithm for managing a replicated log
proposed by Diego Ongaro and John Ousterhout~\cite{ongaro2014search}.
It is equivalent to the state of the art (multi-)Paxos~\cite{lamport1998part} in fault-tolerance and performance.
Consensus algorithms tend to be complex due to the huge problem they try to solve,
and leaving space for interpretation, like Lamport
did when proposing Paxos via a tale~\cite{lamport1998part}, does not help.
The combination of Paxos' complexity and the way it was presented required
Lamport to release an alternative description~\cite{lamport2001paxos}.
This complexity makes Paxos hard to implement in real world applications, which are sometimes
required to relax the model described leading to unproved protocols~\cite{chandra2007paxos}.
The same reason makes it difficult to learn and study for the students and researchers.\\

Ongaro thus identified in \emph{understandability} its ``most important goal''~\cite{ongaro2014consensus}
while describing Raft, breaking it into smaller independant subproblems and cleanly addressing all pieces needed
for a pratical application.
The simplest way to present and understand an interaction based
protocol is visualizing it.

Raft Scope~\cite{raftscope} is a simulator that runs in a browser (Figure~\ref{fig:original}) developed by the very same
author Diego Ongaro.
It simulates the \emph{leader election} and the \emph{log replication}
protocols as described in his original Ph.D. thesis~\cite{ongaro2014consensus}.
This tool is very useful to simulate easy scenarios,
however it does not currently implement all the features of the protocol and lacks
some graphics cues which can be key in the presentation of many corner cases.

In our project we aim to add more features to the protocol implementation and improve the user interface where needed.
In particular, we plan to add functionality to handle cluster membership changes and channel noise.
 The code was also reorganized in order to make it clearer and easier to modify for other users.

Raft Scope is an open-source project currently hosted on GitHub and we plan to
submit a pull request since this could be useful for others.

\newpage